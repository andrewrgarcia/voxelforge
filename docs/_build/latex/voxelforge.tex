%% Generated by Sphinx.
\def\sphinxdocclass{report}
\documentclass[letterpaper,10pt,english]{sphinxmanual}
\ifdefined\pdfpxdimen
   \let\sphinxpxdimen\pdfpxdimen\else\newdimen\sphinxpxdimen
\fi \sphinxpxdimen=.75bp\relax
\ifdefined\pdfimageresolution
    \pdfimageresolution= \numexpr \dimexpr1in\relax/\sphinxpxdimen\relax
\fi
%% let collapsible pdf bookmarks panel have high depth per default
\PassOptionsToPackage{bookmarksdepth=5}{hyperref}

\PassOptionsToPackage{booktabs}{sphinx}
\PassOptionsToPackage{colorrows}{sphinx}

\PassOptionsToPackage{warn}{textcomp}
\usepackage[utf8]{inputenc}
\ifdefined\DeclareUnicodeCharacter
% support both utf8 and utf8x syntaxes
  \ifdefined\DeclareUnicodeCharacterAsOptional
    \def\sphinxDUC#1{\DeclareUnicodeCharacter{"#1}}
  \else
    \let\sphinxDUC\DeclareUnicodeCharacter
  \fi
  \sphinxDUC{00A0}{\nobreakspace}
  \sphinxDUC{2500}{\sphinxunichar{2500}}
  \sphinxDUC{2502}{\sphinxunichar{2502}}
  \sphinxDUC{2514}{\sphinxunichar{2514}}
  \sphinxDUC{251C}{\sphinxunichar{251C}}
  \sphinxDUC{2572}{\textbackslash}
\fi
\usepackage{cmap}
\usepackage[T1]{fontenc}
\usepackage{amsmath,amssymb,amstext}
\usepackage{babel}



\usepackage{tgtermes}
\usepackage{tgheros}
\renewcommand{\ttdefault}{txtt}



\usepackage[Bjarne]{fncychap}
\usepackage{sphinx}

\fvset{fontsize=auto}
\usepackage{geometry}


% Include hyperref last.
\usepackage{hyperref}
% Fix anchor placement for figures with captions.
\usepackage{hypcap}% it must be loaded after hyperref.
% Set up styles of URL: it should be placed after hyperref.
\urlstyle{same}


\usepackage{sphinxmessages}
\setcounter{tocdepth}{1}



\title{VoxelForge}
\date{Aug 15, 2024}
\release{0.1}
\author{Andrew Garcia}
\newcommand{\sphinxlogo}{\vbox{}}
\renewcommand{\releasename}{Release}
\makeindex
\begin{document}

\ifdefined\shorthandoff
  \ifnum\catcode`\=\string=\active\shorthandoff{=}\fi
  \ifnum\catcode`\"=\active\shorthandoff{"}\fi
\fi

\pagestyle{empty}
\sphinxmaketitle
\pagestyle{plain}
\sphinxtableofcontents
\pagestyle{normal}
\phantomsection\label{\detokenize{index::doc}}


\begin{figure}[htbp]
\centering

\sphinxhref{https://github.com/andrewrgarcia/voxelforge}{\sphinxincludegraphics[width=600\sphinxpxdimen]{{banner}.png}}
\end{figure}

\sphinxAtStartPar
VoxelForge is a high\sphinxhyphen{}performance Python package designed for creating and manipulating voxel models, with advanced spatial data structures for ML and deep learning integration.

\sphinxAtStartPar
Contents:

\sphinxstepscope


\chapter{Installation}
\label{\detokenize{installation:installation}}\label{\detokenize{installation::doc}}
\sphinxAtStartPar
\sphinxstylestrong{VoxelForge} can be easily installed using pip. Ensure you have Python and pip installed on your system, then run the following command:

\begin{sphinxVerbatim}[commandchars=\\\{\}]
pip\PYG{+w}{ }install\PYG{+w}{ }VoxelForge
\end{sphinxVerbatim}

\sphinxAtStartPar
This command will download and install VoxelForge and all required dependencies.

\sphinxstepscope


\chapter{Usage}
\label{\detokenize{usage:usage}}\label{\detokenize{usage::doc}}

\section{Basic Voxel Operations}
\label{\detokenize{usage:basic-voxel-operations}}
\sphinxAtStartPar
To start using \sphinxstylestrong{VoxelForge}, import the package and create a \sphinxtitleref{VoxelGrid} instance:

\begin{sphinxVerbatim}[commandchars=\\\{\}]
\PYG{k+kn}{import} \PYG{n+nn}{voxelforge} \PYG{k}{as} \PYG{n+nn}{vf}

\PYG{c+c1}{\PYGZsh{} Create a VoxelGrid and add voxels}
\PYG{n}{grid} \PYG{o}{=} \PYG{n}{vf}\PYG{o}{.}\PYG{n}{VoxelGrid}\PYG{p}{(}\PYG{p}{)}
\PYG{n}{grid}\PYG{o}{.}\PYG{n}{addVoxel}\PYG{p}{(}\PYG{l+m+mi}{1}\PYG{p}{,} \PYG{l+m+mi}{2}\PYG{p}{,} \PYG{l+m+mi}{3}\PYG{p}{)}
\PYG{n}{grid}\PYG{o}{.}\PYG{n}{addVoxel}\PYG{p}{(}\PYG{l+m+mi}{4}\PYG{p}{,} \PYG{l+m+mi}{5}\PYG{p}{,} \PYG{l+m+mi}{6}\PYG{p}{,} \PYG{l+s+s2}{\PYGZdq{}}\PYG{l+s+s2}{String identity}\PYG{l+s+s2}{\PYGZdq{}}\PYG{p}{)}
\PYG{n}{grid}\PYG{o}{.}\PYG{n}{addVoxel}\PYG{p}{(}\PYG{l+m+mi}{7}\PYG{p}{,} \PYG{l+m+mi}{8}\PYG{p}{,} \PYG{l+m+mi}{9}\PYG{p}{,} \PYG{p}{\PYGZob{}}\PYG{l+s+s2}{\PYGZdq{}}\PYG{l+s+s2}{key}\PYG{l+s+s2}{\PYGZdq{}}\PYG{p}{:} \PYG{l+s+s2}{\PYGZdq{}}\PYG{l+s+s2}{value}\PYG{l+s+s2}{\PYGZdq{}}\PYG{p}{\PYGZcb{}}\PYG{p}{)}

\PYG{n}{voxels} \PYG{o}{=} \PYG{n}{grid}\PYG{o}{.}\PYG{n}{getVoxels}\PYG{p}{(}\PYG{p}{)}
\PYG{k}{for} \PYG{n}{voxel} \PYG{o+ow}{in} \PYG{n}{voxels}\PYG{p}{:}
    \PYG{n+nb}{print}\PYG{p}{(}\PYG{l+s+sa}{f}\PYG{l+s+s1}{\PYGZsq{}}\PYG{l+s+s1}{Voxel at (}\PYG{l+s+si}{\PYGZob{}}\PYG{n}{voxel}\PYG{o}{.}\PYG{n}{x}\PYG{l+s+si}{\PYGZcb{}}\PYG{l+s+s1}{, }\PYG{l+s+si}{\PYGZob{}}\PYG{n}{voxel}\PYG{o}{.}\PYG{n}{y}\PYG{l+s+si}{\PYGZcb{}}\PYG{l+s+s1}{, }\PYG{l+s+si}{\PYGZob{}}\PYG{n}{voxel}\PYG{o}{.}\PYG{n}{z}\PYG{l+s+si}{\PYGZcb{}}\PYG{l+s+s1}{) with data }\PYG{l+s+si}{\PYGZob{}}\PYG{n}{voxel}\PYG{o}{.}\PYG{n}{data}\PYG{l+s+si}{\PYGZcb{}}\PYG{l+s+s1}{\PYGZsq{}}\PYG{p}{)}
\end{sphinxVerbatim}


\section{Advanced Graph Features}
\label{\detokenize{usage:advanced-graph-features}}
\sphinxAtStartPar
\sphinxstylestrong{VoxelForge} also supports transforming voxel data into graph structures, useful for graph\sphinxhyphen{}based machine learning models:

\begin{sphinxVerbatim}[commandchars=\\\{\}]
\PYG{n}{graph\PYGZus{}data} \PYG{o}{=} \PYG{n}{grid}\PYG{o}{.}\PYG{n}{toTorchGraph}\PYG{p}{(}\PYG{n}{xDim}\PYG{o}{=}\PYG{l+m+mi}{10}\PYG{p}{,} \PYG{n}{yDim}\PYG{o}{=}\PYG{l+m+mi}{10}\PYG{p}{,} \PYG{n}{zDim}\PYG{o}{=}\PYG{l+m+mi}{10}\PYG{p}{,} \PYG{n}{neighboring\PYGZus{}radius}\PYG{o}{=}\PYG{l+m+mf}{1.0}\PYG{p}{)}
\PYG{n+nb}{print}\PYG{p}{(}\PYG{l+s+s2}{\PYGZdq{}}\PYG{l+s+s2}{Node Features:}\PYG{l+s+s2}{\PYGZdq{}}\PYG{p}{,} \PYG{n}{graph\PYGZus{}data}\PYG{p}{[}\PYG{l+s+s1}{\PYGZsq{}}\PYG{l+s+s1}{x}\PYG{l+s+s1}{\PYGZsq{}}\PYG{p}{]}\PYG{p}{)}
\PYG{n+nb}{print}\PYG{p}{(}\PYG{l+s+s2}{\PYGZdq{}}\PYG{l+s+s2}{Edge Index:}\PYG{l+s+s2}{\PYGZdq{}}\PYG{p}{,} \PYG{n}{graph\PYGZus{}data}\PYG{p}{[}\PYG{l+s+s1}{\PYGZsq{}}\PYG{l+s+s1}{edge\PYGZus{}index}\PYG{l+s+s1}{\PYGZsq{}}\PYG{p}{]}\PYG{p}{)}
\end{sphinxVerbatim}

\sphinxstepscope


\chapter{API Reference}
\label{\detokenize{api_reference:api-reference}}\label{\detokenize{api_reference::doc}}
\sphinxAtStartPar
The API reference details all classes, methods, and their functions:
\index{module@\spxentry{module}!VoxelForge@\spxentry{VoxelForge}}\index{VoxelForge@\spxentry{VoxelForge}!module@\spxentry{module}}

\section{Voxel}
\label{\detokenize{api_reference:module-VoxelForge.Voxel}}\label{\detokenize{api_reference:voxel}}\label{\detokenize{api_reference:module-VoxelForge}}\index{module@\spxentry{module}!VoxelForge.Voxel@\spxentry{VoxelForge.Voxel}}\index{VoxelForge.Voxel@\spxentry{VoxelForge.Voxel}!module@\spxentry{module}}

\section{VoxelGrid}
\label{\detokenize{api_reference:module-VoxelForge.VoxelGrid}}\label{\detokenize{api_reference:voxelgrid}}\index{module@\spxentry{module}!VoxelForge.VoxelGrid@\spxentry{VoxelForge.VoxelGrid}}\index{VoxelForge.VoxelGrid@\spxentry{VoxelForge.VoxelGrid}!module@\spxentry{module}}
\sphinxstepscope


\chapter{Examples}
\label{\detokenize{examples:examples}}\label{\detokenize{examples::doc}}

\section{Example 1: Basic Voxel Manipulation}
\label{\detokenize{examples:example-1-basic-voxel-manipulation}}
\sphinxAtStartPar
This example demonstrates how to create a VoxelGrid and add voxels with various identities:

\begin{sphinxVerbatim}[commandchars=\\\{\}]
\PYG{c+c1}{\PYGZsh{} Import the VoxelForge package}
\PYG{k+kn}{import} \PYG{n+nn}{voxelforge} \PYG{k}{as} \PYG{n+nn}{vf}

\PYG{c+c1}{\PYGZsh{} Create a new VoxelGrid instance}
\PYG{n}{grid} \PYG{o}{=} \PYG{n}{vf}\PYG{o}{.}\PYG{n}{VoxelGrid}\PYG{p}{(}\PYG{p}{)}

\PYG{c+c1}{\PYGZsh{} Add several voxels with default and custom identities}
\PYG{n}{grid}\PYG{o}{.}\PYG{n}{addVoxel}\PYG{p}{(}\PYG{l+m+mi}{1}\PYG{p}{,} \PYG{l+m+mi}{2}\PYG{p}{,} \PYG{l+m+mi}{3}\PYG{p}{)}  \PYG{c+c1}{\PYGZsh{} Default identity (integer)}
\PYG{n}{grid}\PYG{o}{.}\PYG{n}{addVoxel}\PYG{p}{(}\PYG{l+m+mi}{4}\PYG{p}{,} \PYG{l+m+mi}{5}\PYG{p}{,} \PYG{l+m+mi}{6}\PYG{p}{,} \PYG{l+s+s2}{\PYGZdq{}}\PYG{l+s+s2}{Colorful Voxel}\PYG{l+s+s2}{\PYGZdq{}}\PYG{p}{)}  \PYG{c+c1}{\PYGZsh{} String identity}
\PYG{n}{grid}\PYG{o}{.}\PYG{n}{addVoxel}\PYG{p}{(}\PYG{l+m+mi}{7}\PYG{p}{,} \PYG{l+m+mi}{8}\PYG{p}{,} \PYG{l+m+mi}{9}\PYG{p}{,} \PYG{p}{\PYGZob{}}\PYG{l+s+s2}{\PYGZdq{}}\PYG{l+s+s2}{key}\PYG{l+s+s2}{\PYGZdq{}}\PYG{p}{:} \PYG{l+s+s2}{\PYGZdq{}}\PYG{l+s+s2}{value}\PYG{l+s+s2}{\PYGZdq{}}\PYG{p}{\PYGZcb{}}\PYG{p}{)}  \PYG{c+c1}{\PYGZsh{} Dictionary identity}
\PYG{n}{grid}\PYG{o}{.}\PYG{n}{addVoxel}\PYG{p}{(}\PYG{l+m+mi}{10}\PYG{p}{,} \PYG{l+m+mi}{11}\PYG{p}{,} \PYG{l+m+mi}{12}\PYG{p}{,} \PYG{l+m+mf}{3.14159}\PYG{p}{)}  \PYG{c+c1}{\PYGZsh{} Floating\PYGZhy{}point identity}

\PYG{c+c1}{\PYGZsh{} Print out the voxel information}
\PYG{n}{voxels} \PYG{o}{=} \PYG{n}{grid}\PYG{o}{.}\PYG{n}{getVoxels}\PYG{p}{(}\PYG{p}{)}
\PYG{k}{for} \PYG{n}{voxel} \PYG{o+ow}{in} \PYG{n}{voxels}\PYG{p}{:}
    \PYG{n+nb}{print}\PYG{p}{(}\PYG{l+s+sa}{f}\PYG{l+s+s1}{\PYGZsq{}}\PYG{l+s+s1}{Voxel at (}\PYG{l+s+si}{\PYGZob{}}\PYG{n}{voxel}\PYG{o}{.}\PYG{n}{x}\PYG{l+s+si}{\PYGZcb{}}\PYG{l+s+s1}{, }\PYG{l+s+si}{\PYGZob{}}\PYG{n}{voxel}\PYG{o}{.}\PYG{n}{y}\PYG{l+s+si}{\PYGZcb{}}\PYG{l+s+s1}{, }\PYG{l+s+si}{\PYGZob{}}\PYG{n}{voxel}\PYG{o}{.}\PYG{n}{z}\PYG{l+s+si}{\PYGZcb{}}\PYG{l+s+s1}{) with data: }\PYG{l+s+si}{\PYGZob{}}\PYG{n}{voxel}\PYG{o}{.}\PYG{n}{data}\PYG{l+s+si}{\PYGZcb{}}\PYG{l+s+s1}{\PYGZsq{}}\PYG{p}{)}
\end{sphinxVerbatim}


\section{Example 2: Using Octrees for Spatial Indexing}
\label{\detokenize{examples:example-2-using-octrees-for-spatial-indexing}}
\sphinxAtStartPar
Here we show how to initialize an Octree, insert points, and locate nodes:

\begin{sphinxVerbatim}[commandchars=\\\{\}]
\PYG{c+c1}{\PYGZsh{} Import the VoxelForge package and numpy for handling coordinates}
\PYG{k+kn}{import} \PYG{n+nn}{voxelforge} \PYG{k}{as} \PYG{n+nn}{vf}
\PYG{k+kn}{import} \PYG{n+nn}{numpy} \PYG{k}{as} \PYG{n+nn}{np}

\PYG{c+c1}{\PYGZsh{} Initialize an Octree with a specific origin, size, and maximum depth}
\PYG{n}{origin} \PYG{o}{=} \PYG{n}{np}\PYG{o}{.}\PYG{n}{array}\PYG{p}{(}\PYG{p}{[}\PYG{l+m+mf}{0.0}\PYG{p}{,} \PYG{l+m+mf}{0.0}\PYG{p}{,} \PYG{l+m+mf}{0.0}\PYG{p}{]}\PYG{p}{)}
\PYG{n}{size} \PYG{o}{=} \PYG{l+m+mf}{50.0}
\PYG{n}{max\PYGZus{}depth} \PYG{o}{=} \PYG{l+m+mi}{4}
\PYG{n}{octree} \PYG{o}{=} \PYG{n}{vf}\PYG{o}{.}\PYG{n}{Octree}\PYG{p}{(}\PYG{n}{origin}\PYG{p}{,} \PYG{n}{size}\PYG{p}{,} \PYG{n}{max\PYGZus{}depth}\PYG{p}{)}

\PYG{c+c1}{\PYGZsh{} Insert points into the Octree}
\PYG{n}{points} \PYG{o}{=} \PYG{p}{[}
    \PYG{n}{np}\PYG{o}{.}\PYG{n}{array}\PYG{p}{(}\PYG{p}{[}\PYG{l+m+mf}{5.0}\PYG{p}{,} \PYG{l+m+mf}{5.0}\PYG{p}{,} \PYG{l+m+mf}{5.0}\PYG{p}{]}\PYG{p}{)}\PYG{p}{,}
    \PYG{n}{np}\PYG{o}{.}\PYG{n}{array}\PYG{p}{(}\PYG{p}{[}\PYG{l+m+mf}{15.0}\PYG{p}{,} \PYG{l+m+mf}{15.0}\PYG{p}{,} \PYG{l+m+mf}{15.0}\PYG{p}{]}\PYG{p}{)}\PYG{p}{,}
    \PYG{n}{np}\PYG{o}{.}\PYG{n}{array}\PYG{p}{(}\PYG{p}{[}\PYG{l+m+mf}{35.0}\PYG{p}{,} \PYG{l+m+mf}{35.0}\PYG{p}{,} \PYG{l+m+mf}{35.0}\PYG{p}{]}\PYG{p}{)}\PYG{p}{,}
    \PYG{n}{np}\PYG{o}{.}\PYG{n}{array}\PYG{p}{(}\PYG{p}{[}\PYG{l+m+mf}{45.0}\PYG{p}{,} \PYG{l+m+mf}{45.0}\PYG{p}{,} \PYG{l+m+mf}{45.0}\PYG{p}{]}\PYG{p}{)}
\PYG{p}{]}

\PYG{k}{for} \PYG{n}{point} \PYG{o+ow}{in} \PYG{n}{points}\PYG{p}{:}
    \PYG{n}{octree}\PYG{o}{.}\PYG{n}{insert\PYGZus{}point}\PYG{p}{(}\PYG{n}{point}\PYG{p}{)}

\PYG{c+c1}{\PYGZsh{} Locate a specific point and print its information}
\PYG{n}{target\PYGZus{}point} \PYG{o}{=} \PYG{n}{np}\PYG{o}{.}\PYG{n}{array}\PYG{p}{(}\PYG{p}{[}\PYG{l+m+mf}{5.0}\PYG{p}{,} \PYG{l+m+mf}{5.0}\PYG{p}{,} \PYG{l+m+mf}{5.0}\PYG{p}{]}\PYG{p}{)}
\PYG{n}{leaf\PYGZus{}node} \PYG{o}{=} \PYG{n}{octree}\PYG{o}{.}\PYG{n}{locate\PYGZus{}leaf\PYGZus{}node}\PYG{p}{(}\PYG{n}{target\PYGZus{}point}\PYG{p}{)}
\PYG{k}{if} \PYG{n}{leaf\PYGZus{}node}\PYG{p}{:}
    \PYG{n+nb}{print}\PYG{p}{(}\PYG{l+s+sa}{f}\PYG{l+s+s2}{\PYGZdq{}}\PYG{l+s+s2}{Leaf node found at }\PYG{l+s+si}{\PYGZob{}}\PYG{n}{leaf\PYGZus{}node}\PYG{o}{.}\PYG{n}{get\PYGZus{}point}\PYG{p}{(}\PYG{p}{)}\PYG{l+s+si}{\PYGZcb{}}\PYG{l+s+s2}{\PYGZdq{}}\PYG{p}{)}
\PYG{k}{else}\PYG{p}{:}
    \PYG{n+nb}{print}\PYG{p}{(}\PYG{l+s+s2}{\PYGZdq{}}\PYG{l+s+s2}{No leaf node found at the specified location.}\PYG{l+s+s2}{\PYGZdq{}}\PYG{p}{)}
\end{sphinxVerbatim}

\sphinxstepscope


\chapter{Frequently Asked Questions}
\label{\detokenize{faq:frequently-asked-questions}}\label{\detokenize{faq::doc}}\begin{enumerate}
\sphinxsetlistlabels{\arabic}{enumi}{enumii}{}{.}%
\item {} 
\sphinxAtStartPar
\sphinxstylestrong{What is VoxelForge?}

\sphinxAtStartPar
VoxelForge is a Python package for efficient voxel and mesh model creation with a focus on integration in machine learning workflows.

\item {} 
\sphinxAtStartPar
\sphinxstylestrong{How can I contribute to VoxelForge?}

\sphinxAtStartPar
Contributions can be made via our GitHub repository, whether as feature suggestions, bug reports, or pull requests.

\end{enumerate}


\chapter{Indices and tables}
\label{\detokenize{index:indices-and-tables}}\begin{itemize}
\item {} 
\sphinxAtStartPar
\DUrole{xref,std,std-ref}{genindex}

\item {} 
\sphinxAtStartPar
\DUrole{xref,std,std-ref}{modindex}

\item {} 
\sphinxAtStartPar
\DUrole{xref,std,std-ref}{search}

\end{itemize}


\renewcommand{\indexname}{Python Module Index}
\begin{sphinxtheindex}
\let\bigletter\sphinxstyleindexlettergroup
\bigletter{v}
\item\relax\sphinxstyleindexentry{VoxelForge}\sphinxstyleindexpageref{api_reference:\detokenize{module-VoxelForge}}
\item\relax\sphinxstyleindexentry{VoxelForge.Voxel}\sphinxstyleindexpageref{api_reference:\detokenize{module-VoxelForge.Voxel}}
\item\relax\sphinxstyleindexentry{VoxelForge.VoxelGrid}\sphinxstyleindexpageref{api_reference:\detokenize{module-VoxelForge.VoxelGrid}}
\end{sphinxtheindex}

\renewcommand{\indexname}{Index}
\printindex
\end{document}